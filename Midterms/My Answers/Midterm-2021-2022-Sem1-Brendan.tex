\documentclass{article}
\usepackage{fullpage} %for 1-inch margins
\usepackage{fancyvrb} %For code blocks
\usepackage{lmodern}
\fvset{tabsize=4} %Set indentation to 4
\usepackage{enumitem} %for itemize alphabetical
\setlength\parindent{0pt} %no indent

\title{Midterm CS2040C Sem 1 2021-2022}
\author{Brendan Cheong Ern Jie}
\date {08/02/2022}

\begin{document}
\maketitle

\section{Question 01}
\begin{enumerate}[label=(\arabic*)]
\item What will be printed if we run that code? (any answer with the first two most significant digits correct will be accepted)

[$N \times \log_2(N) \times 2$ \rightarrow 2048 \times 11 \times 2 = 45056]

\item What is the time complexity of the code above in terms of N?

[O(N \log(N))]

\item If N in line 5 is changed to 1000000 (1 Million), do you think your code can run in less
than 1s if your computer can do 100 Million basic operations in 1s? Write Y/N in the
box:

[N]

\item If the comment "//" in line 8 is removed so that line 8 is part of the code, will the time
complexity that you have answered in part 2 changes? Write Y/N in the box:

[N]

\item  If line 10 is changed to for (int k = 0; k $<$ N; ++k), the time complexity of the code
above in terms of N changes into O(5) 

[$N^2\log(N)$]

\end{enumerate}

\section{Question 02}
\begin{enumerate}[label=(\arabic*)]

\item \begin{verbatim}
#include <1>
\end{verbatim}

[algorithm]

\item \begin{verbatim} 
for (auto 2name : names)
\end{verbatim}

[\&]

\item \begin{verbatim} 
std::sort(names.3(), names.end(), [](const std::string &a, const std::string &b) {}
\end{verbatim}

[begin]

\item \begin{verbatim}
if (a.length() 4 b.length())
\end{verbatim}

[$!=$]

\item \begin{verbatim}
return a.length() 5 b.length();
\end{verbatim}

[$<$]

\end{enumerate}

\section{Question 03}

\begin{enumerate}[label=(\arabic*)]
\item Choose the Data Structure to be used

\begin{verbatim}
std::vector
\end{verbatim}

\item How do you implement add(v)

[using the vector method \begin{verbatim} push_back(v)\end{verbatim} with the input being the element. This method
implementation has a O(1) time complexity]

\item How do you implement count()

[using the vector method size() with the output being the size of the vector or the number of items in the ADT. The time
complexity will be O(1)]

\item How do you implement returnAnyWithEqualProbability()

[let the index be: \begin{verbatim}
int index = rand() % size();
\end{verbatim}
where this will randomly choose an index within the size of the ADT. Then once the index is found, use the STL vector in built \begin{verbatim} 
vector.remove(index);
\end{verbatim}
Finding the random index is O(1) while removing the element is O(N)
]

\end{enumerate}

\section{Question 04}

\item What time complexity?

[O($N^2$)]

\section{Question 05}

\item What is the time complexity?

[O(N)]

\section{Question 06}

\item What is the time complexity?

[O($N\log(N)$)]

\section{Question 07}

\item What is the time complexity?

[O($N^2$)]

\section{Question 08}

\item explain how newSort() works and why it can sort properly (HARD)

[Now the Quicksort doesn't sort the whole array entirely, leaving some portions of it unsorted. Thus, with the addition
of insertionSort, it will sort the rest of the array in O(K) times, meaning the time complexity is
$O((N - K)\log(N- K) + K)$]

\section{Question 09}
\begin{enumerate}
\item What is the time complexity?

[O($N^2$)]

\item What is the time complexity?

[O($N \log(N)$)]
\end{enumerate}

\section{Question 10}

\item What is the expected time complexity of newSort() in terms of K (that can range from 1 to N) and N? (HARD)

[The expected time complexity of newSort() in terms of K is $O((N-K)\log(N-K) + K^2$. This is because the worst
case scenario to sort K elements for insertion sort is $O(K^2)$ while the quick sort sorts the rest of the elements that K missed out which is $(N-K)\log(N-K)$ so combined together its $O((N-K)\log(N-K) + K^2$]

\section{Question 11}
\begin{enumerate}[label=(\arabic*)]
\item Fill In the Blanks

[1]
\item Fill In the Blanks

[3]
\item Fill In the Blanks

[9]
\item Fill In the Blanks

[7]
\item Fill In the Blanks

[N]
\end{enumerate}

\section{Question 12}
\begin{verbatim}
#include <iostream>
#include <algorithm> // blanks
#include <vector>
#include <utility>
#include <unordered_map>

using namespace std;
typedef pair<int, int> pii;
typedef pair<int, pii> pip;
typedef unordered_map<int, pii> mip; // {value, {firstOccur, count}}

int main() {
    int N, K; cin >> N >> K; // 1 <= N <= 1M, 1 <= K <= N

    mip map;
    for (int i  = 0; i < N; ++i) { // O(N) here
        int v; cin >> v;
        if (map.find(v) != map.end()) {
            ++map[v].second;
        } else {
            map[v] = {i, 1};
        }
    }

    vector<pip> A;
    for (auto &e : map) {
        A.push_back({e.first, e.second});
    }

    // sort
    sort(A.begin(), A.end(), [](pip &a, pip &b){ // {value, {firstOccur, count}}
        if (a.second.second != b.second.second) {
            return a.second.second > b.second.second;
        } else {
            return a.second.first < b.second.first;
        }
    });

    for (auto a : A) {
        for (auto i = 0; i < a.second.second; ++i)
            cout << a.first << " ";
    }
    cout << endl;
    return 0;
}
\end{verbatim}
[the overall time complexity of my C++ code above is $O(N + N2 \log(N2))$, where N2is the number of unique numbers in N inputs]
\end{document}